%%
%% Automatically generated file from Doconce source
%% (https://github.com/hplgit/doconce/)
%%
% #ifdef PTEX2TEX_EXPLANATION
%%
%% The file follows the ptex2tex extended LaTeX format, see
%% ptex2tex: http://code.google.com/p/ptex2tex/
%%
%% Run
%%      ptex2tex myfile
%% or
%%      doconce ptex2tex myfile
%%
%% to turn myfile.p.tex into an ordinary LaTeX file myfile.tex.
%% (The ptex2tex program: http://code.google.com/p/ptex2tex)
%% Many preprocess options can be added to ptex2tex or doconce ptex2tex
%%
%%      ptex2tex -DMINTED -DPALATINO -DA6PAPER -DLATEX_HEADING=traditional myfile
%%      doconce ptex2tex myfile -DLATEX_HEADING=titlepage envir=minted
%%
%% ptex2tex will typeset code environments according to a global or local
%% .ptex2tex.cfg configure file. doconce ptex2tex will typeset code
%% according to options on the command line (just type doconce ptex2tex to
%% see examples). If doconce ptex2tex has envir=minted, it enables the
%% minted style without needing -DMINTED.
% #endif

% #ifndef LATEX_STYLE
% #define LATEX_STYLE "std"
% #endif

% #ifndef LATEX_HEADING
% #define LATEX_HEADING "doconce_heading"
% #endif

% #ifndef PREAMBLE
% #if LATEX_HEADING == "Springer_collection"
% #undef PREAMBLE
% #else
% #define PREAMBLE
% #endif
% #endif


% #ifdef PREAMBLE
%-------------------- begin preamble ----------------------
% #if LATEX_STYLE == "std"

\documentclass[%
twoside,                 % oneside: electronic viewing, twoside: printing
final,                   % or draft (marks overfull hboxes, figures with paths)
10pt]{article}

% #elif LATEX_STYLE == "Springer_lncse"
% Style: Lecture Notes in Computational Science and Engineering (Springer)
\documentclass[envcountsect,open=right]{lncse}
\pagestyle{headings}
% #elif LATEX_STYLE == "Springer_T2"
% Style: T2 (Springer)
\documentclass[graybox,sectrefs,envcountresetchap,open=right]{svmono}
\usepackage{t2}
\special{papersize=193mm,260mm}
% #elif LATEX_STYLE == "Springer_llcse"
% Style: Lecture Notes in Computer Science (Springer)
\documentclass[oribib]{llncs}
% #elif LATEX_STYLE == "Koma_Script"
% Style: Koma-Script
\documentclass[10pt]{scrartcl}
% #elif LATEX_STYLE == "siamltex"
% Style: SIAM LaTeX2e
\documentclass[leqno]{siamltex}
% #elif LATEX_STYLE == "siamltexmm"
% Style: SIAM LaTeX2e multimedia
\documentclass[leqno]{siamltexmm}
% #endif

\listfiles               % print all files needed to compile this document

% #ifdef A4PAPER
\usepackage[a4paper]{geometry}
% #endif
% #ifdef A6PAPER
% a6paper is suitable for mobile devices
\usepackage[%
  a6paper,
  text={90mm,130mm},
  inner={5mm},           % inner margin (two sided documents)
  top=5mm,
  headsep=4mm
  ]{geometry}
% #endif

\usepackage{relsize,epsfig,makeidx,color,setspace,amsmath,amsfonts}
\usepackage[table]{xcolor}
\usepackage{bm,microtype}

\usepackage{ptex2tex}

% #ifdef XELATEX
% xelatex settings
\usepackage{fontspec}
\usepackage{xunicode}
\defaultfontfeatures{Mapping=tex-text} % To support LaTeX quoting style
\defaultfontfeatures{Ligatures=TeX}
\setromanfont{Kinnari}
% Examples of font types (Ubuntu): Gentium Book Basic (Palatino-like),
% Liberation Sans (Helvetica-like), Norasi, Purisa (handwriting), UnDoum
% #else
\usepackage[T1]{fontenc}
%\usepackage[latin1]{inputenc}
\usepackage[utf8]{inputenc}
% #ifdef HELVETICA
% Set helvetica as the default font family:
\RequirePackage{helvet}
\renewcommand\familydefault{phv}
% #endif
% #ifdef PALATINO
% Set palatino as the default font family:
\usepackage[sc]{mathpazo}    % Palatino fonts
\linespread{1.05}            % Palatino needs extra line spread to look nice
% #endif
% #endif
\usepackage{lmodern}         % Latin Modern fonts derived from Computer Modern

% Hyperlinks in PDF:
\definecolor{linkcolor}{rgb}{0,0,0.4}
\usepackage[%
    colorlinks=true,
    linkcolor=linkcolor,
    urlcolor=linkcolor,
    citecolor=black,
    filecolor=black,
    %filecolor=blue,
    pdfmenubar=true,
    pdftoolbar=true,
    bookmarksdepth=3   % Uncomment (and tweak) for PDF bookmarks with more levels than the TOC
            ]{hyperref}
%\hyperbaseurl{}   % hyperlinks are relative to this root

\setcounter{tocdepth}{2}  % number chapter, section, subsection

% --- end of definitions of admonition environments ---

% prevent orhpans and widows
\clubpenalty = 10000
\widowpenalty = 10000

% #ifndef SECTION_HEADINGS
% #define SECTION_HEADINGS "std"
% #else
% http://www.ctex.org/documents/packages/layout/titlesec.pdf
\usepackage[compact]{titlesec}  % reduce the spacing above/below the heading
% #endif
% #if SECTION_HEADINGS == "blue"
% --- section/subsection headings with blue color ---
\definecolor{seccolor}{cmyk}{.9,.5,0,.35}  % siamltexmm.sty section color
\titleformat{name=\section}
{\color{seccolor}\normalfont\Large\bfseries}
{\color{seccolor}\thesection}{1em}{}
\titleformat{name=\subsection}
{\color{seccolor}\normalfont\large\bfseries}
{\color{seccolor}\thesubsection}{1em}{}
\titleformat{name=\paragraph}[runin]
{\color{seccolor}\normalfont\normalsize\bfseries}
{}{}{\indent}
% #ifdef FANCY_HEADER
% let the header have a thick gray hrule with section and page in blue above
\renewcommand{\headrulewidth}{0.4pt}
\renewcommand{\headrule}{{\color{gray!50}%
\hrule width\headwidth height\headrulewidth \vskip-\headrulewidth}}
\fancyhead[LE,RO]{{\color{seccolor}\rightmark}} %section
\fancyhead[RE,LO]{{\color{seccolor}\thepage}}
% #endif
% #elif SECTION_HEADINGS == "strongblue"
% --- section/subsection headings with a strong blue color ---
\definecolor{seccolor}{rgb}{0.2,0.2,0.8}
\titleformat{name=\section}
{\color{seccolor}\normalfont\Large\bfseries}
{\color{seccolor}\thesection}{1em}{}
\titleformat{name=\subsection}
{\color{seccolor}\normalfont\large\bfseries}
{\color{seccolor}\thesubsection}{1em}{}
\titleformat{name=\paragraph}[runin]
{\color{seccolor}\normalfont\normalsize\bfseries}
{}{}{\indent}
% #elif SECTION_HEADINGS == "gray"
% --- section/subsection headings with white text on gray background ---
\titleformat{name=\section}[block]
  {\sffamily\Large}{}{0pt}{\colorsection}
\titlespacing*{\section}{0pt}{\baselineskip}{\baselineskip}

\newcommand{\colorsection}[1]{%
  \colorbox{gray!50}{{\color{white}\thesection\ #1}}}

\titleformat{name=\subsection}[block]
  {\sffamily\large}{}{0pt}{\colorsubsection}
\titlespacing*{\subsection}{0pt}{\baselineskip}{\baselineskip}

\newcommand{\colorsubsection}[1]{%
  \colorbox{gray!50}{{\color{white}\thesubsection\ #1}}}
% #elif SECTION_HEADINGS == "gray-wide"
% --- section/subsection headings with white text on wide gray background ---
\titleformat{name=\section}[block]
  {\sffamily\Large}{}{0pt}{\colorsection}
\titlespacing*{\section}{0pt}{\baselineskip}{\baselineskip}

\newcommand{\colorsection}[1]{%
  \colorbox{gray!50}{\parbox{\dimexpr\textwidth-2\fboxsep}%
           {\color{white}\thesection\ #1}}}

\titleformat{name=\subsection}[block]
  {\sffamily\large}{}{0pt}{\colorsubsection}
\titlespacing*{\subsection}{0pt}{\baselineskip}{\baselineskip}

\newcommand{\colorsubsection}[1]{%
  \colorbox{gray!50}{\parbox{\dimexpr\textwidth-2\fboxsep}%
           {\color{white}\thesubsection\ #1}}}
% #endif

% #ifdef COLORED_TABLE_ROWS
% color every two table rows
\let\oldtabular\tabular
\let\endoldtabular\endtabular
% #if COLORED_TABLE_ROWS not in ("gray", "blue")
% #define COLORED_TABLE_ROWS gray
% #endif
% #else
% #define COLORED_TABLE_ROWS no
% #endif
% #if COLORED_TABLE_ROWS == "gray"
\definecolor{rowgray}{gray}{0.9}
\renewenvironment{tabular}{\rowcolors{2}{white}{rowgray}%
\oldtabular}{\endoldtabular}
% #elif COLORED_TABLE_ROWS == "blue"
\definecolor{appleblue}{rgb}{0.93,0.95,1.0}  % Apple blue
\renewenvironment{tabular}{\rowcolors{2}{white}{appleblue}%
\oldtabular}{\endoldtabular}
% #endif


% --- end of standard preamble for documents ---


% insert custom LaTeX commands...

\raggedbottom
\makeindex

%-------------------- end preamble ----------------------

\begin{document}

% #endif


% ------------------- main content ----------------------



% ----------------- title -------------------------

% #if LATEX_HEADING == "traditional"

% #if SECTION_HEADINGS in ("blue", "strongblue")
\title{{\color{seccolor} Epleband 19/1-2013}}
% #else
\title{Epleband 19/1-2013}
% #endif

% #elif LATEX_HEADING == "titlepage"

\thispagestyle{empty}
\hbox{\ \ }
\vfill
\begin{center}
{\huge{\bfseries{
\begin{spacing}{1.25}
% #if SECTION_HEADINGS in ("blue", "strongblue")
{\color{seccolor}\rule{\linewidth}{0.5mm}} \\[0.4cm]
{\color{seccolor}Epleband 19/1-2013}
\\[0.4cm] {\color{seccolor}\rule{\linewidth}{0.5mm}} \\[1.5cm]
% #else
{\rule{\linewidth}{0.5mm}} \\[0.4cm]
{Epleband 19/1-2013}
\\[0.4cm] {\rule{\linewidth}{0.5mm}} \\[1.5cm]
% #endif
\end{spacing}
}}}

% #elif LATEX_HEADING == "Springer_collection"
\title*{Epleband 19/1-2013}
% Short version of title:
\titlerunning{Epleband 19/1-2013}

% #elif LATEX_HEADING == "beamer"
\title{Epleband 19/1-2013}
% #else
\thispagestyle{empty}

\begin{center}
{\LARGE\bf
\begin{spacing}{1.25}
Epleband 19/1-2013
\end{spacing}
}
\end{center}
% #endif

% AUTHOR: Jan, Kristian, Erik, Joakim, Tore Magnus, Hans Petter


% ----------------- author(s) -------------------------
% #if LATEX_HEADING == "traditional"
\author{Hans Petter++}

% #elif LATEX_HEADING == "titlepage"
\vspace{1.3cm}

    {\Large\textsf{Hans Petter++${}^{}$}}\\ [3mm]
    
\ \\ [2mm]

% #elif LATEX_HEADING == "Springer_collection"

\author{Hans Petter++}
% Short version of authors:
%\authorrunning{...}
\institute{Hans Petter++}

% #elif LATEX_HEADING == "beamer"
\author{Hans Petter++\inst{}}
\institute{}
% #else

\begin{center}
{\bf Hans Petter++${}^{}$} \\ [0mm]
\end{center}

\begin{center}
% List of all institutions:
\end{center}
% #endif
% ----------------- end author(s) -------------------------


% #if LATEX_HEADING == "traditional"
\date{Jan 15, 2014}
\maketitle
% #elif LATEX_HEADING == "beamer"
\date{Jan 15, 2014
% <titlepage figure>
}
% #elif LATEX_HEADING == "titlepage"

\ \\ [10mm]
{\large\textsf{Jan 15, 2014}}

\end{center}
\vfill
\clearpage

% #else
\begin{center}
Jan 15, 2014
\end{center}

\vspace{1cm}

% #endif


% #if LATEX_HEADING != "beamer"


\vspace{1cm} % after toc'
% #endif





% !split
\paragraph{Proud Mary.}
intro, vers, refreng, intro, vers, refreng, intro, vers m/solo, refreng,
intro, vers, refreng, refreng, ...

\textbf{Chords:} \emph{intro}: \code{G E G E G E D C C6 C D A};
\emph{vers}: \code{A E F#,}; \emph{refreng}: \code{A} (med to ganger \emph{rollin' on a river})




% !split
\paragraph{Kokken Tor.}
4x vers, refreng, 2x vers, refreng, 2x vers, refreng, refreng m/solo

\textbf{Chords:} \emph{vers}: \code{D Bm D Bm A D F#m A};
\emph{refreng}: \code{3x A E F#m D}, \code{A E F#m A}





% !split
\paragraph{Black Magick Woman.}
intro, vers m/solo, 2x vers m/sang, vers m/solo, vers m/sang, solo gitar, solo trompet, solo gitar, solo trompet, 2x solo gitar.

\textbf{Chords:} \code{Dm Am Dm Gm Dm A Dm}






% !split
\paragraph{Wish You Were Here.}
1x intro, 1x intro m/solo (Joakim), 2x vers, 2x intro m/solo (Joakim), refreng,
gjenta intro med div solo (HP)


\textbf{Chords:} \emph{intro}: \code{2x Em7 G, 2x Em7 A, G};
\emph{vers}: \code{C D Am G D C Am G}; \emph{refreng}: \code{C D Am G C Am G}





% !split
\paragraph{Johnny B Goode.}
intro v/Joakim, vers, refreng, vers, refreng, vers m/solo (HP), refreng,
vers, refreng.

\textbf{Chords:} \emph{vers/refreng}: \code{A D A E A}

Husk: tighte gitarer som følger trommene.


% !split
\paragraph{Wonderful Tonight.}
2x intro gitar, vers, 2x intro gitar, vers, bro, 2x intro trompet, vers m/solo gitar, 2x intro trompet, vers

\textbf{Chords:} \emph{intro}: \code{G D/F# C D G D/F# C D};
\emph{vers}:
\code{C D G D/F# C}, \code{D C D G Bm/F# Em C D G};
\emph{bro}: \code{C D G Bm/F# Em C D C D G}




% !split
\paragraph{Wild Horses.}
1x intro, vers, refreng, vers, refreng, vers m/solo, refreng

\textbf{Chords:} \emph{intro}: \code{G Am7  G Gsus Am7 G};
\emph{vers}: \code{Bm G Gsus}, \code{Bm G Gsus}, \code{Am G C-D}, \code{G Gsus G D Dsus2 D C};
\emph{refreng}: \code{Am G C-D G F G-C}, \code{Bm}, \code{Am G C-D G F G-C}
(NB: ingen bro etter 2.refreng!)





% !split
\paragraph{Brown Sugar.}
4x intro1, 2x intro2, vers, refreng, 2x intro2, vers, refreng, 4x intro2 (\code{Eb-C...}) m/solo, refreng, vers, refreng, deretter gjentagelse av chords i refrenget (\code{G C}).

\textbf{Chords:} \emph{intro1}: \code{C-G C-Csus}; \emph{intro2}: \code{Eb-Eb9-Eb}, \code{C-Csus}, \code{Ab Bb C};
\emph{vers}: \code{C F C Bb C (Bb)}; \emph{refreng}: \code{G C G C}

Husk: tighte gitarer som følger trommene.




% !split
% Compact title combined with compact info
\paragraph{Sensitive Kind.}
intro, 2x vers, solo, vers.

\textbf{Chords:} \emph{intro}: \code{Gm Cm D7 Cm}, \emph{vers}: \code{Gm Cm D#7-D7 Gm}





% !split
% Compact title combined with compact info
\paragraph{Hey Joe.}
% !split
\paragraph{I Saw Her Standing There.}
intro, 2x vers, bridge, vers, solo, bridge, vers

\paragraph{Chords:}
\emph{vers}: \code{C7 F7 C7 G7 C C7 F7 Fm7/Ab C7 G7 C7}; \emph{bridge}: \code{F7 G7 F7}









% !split
% Compact title combined with compact info
\paragraph{Fortunate Son.}
% vers, ...


\textbf{Chords:} \emph{vers}: \code{F# E B F# E B F#}; \emph{refreng}: \code{2x F# C# B F#}





% !split
% Compact title combined with compact info
\paragraph{Back In The USSR.}
intro, 2x vers+refreng, bridge, solo (vers), refreng, bridge, vers, refreng.

\textbf{Chords:} \emph{vers}: \code{A D C D}; \emph{refreng}: \code{A C D A D-Eb-E},
\emph{bridge}: \code{D A D D/C# D/C D/B E D A D-Eb-E}





% !split
% Compact title combined with compact info
\paragraph{Slave.}
3x vers+refreng, bridge, vers+refreng.

\textbf{Chords:} \emph{vers}: \code{Em}; \emph{refreng}: \code{Am Em-Am-Em}






% ------------------- end of main content ---------------


% #ifdef PREAMBLE
\printindex

\end{document}
% #endif

